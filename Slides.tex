\documentclass{beamer}
\usetheme{Warsaw}
\title[Introduction to UrWeb]{UrWeb\\Functional. Pure. Principled. Eager. Slightly mad.}
\author{Sean Chalmers}
\date{Sun 26 Apr 2015 14:18:00 AEST}
\begin{document}

\begin{frame}
\titlepage
\end{frame}


\begin{frame}{Introduction}
Brief introduction to the Ur language and its current incarnation in Ur/Web.

\begin{itemize}
\item Going Badless
\item Simple app walkthrough
\item XHTML
\item SQL
\item JavaScript ... sort of
\item Type System
\end{itemize}
\end{frame}

\section{Going Badless}
\subsection{Ur}

\begin{frame}{Ur}
\textbf{Ur} is a relative of Haskell and ML.

\begin{itemize}
\item Pure
\item Statically typed
\item Strict
\item Type level programming
\item Row types
\item Type classes too!
\end{itemize}
\end{frame}

\subsection{Ur/Web}
\begin{frame}{Ur/Web}

\begin{itemize}
\item \textbf{Ur} is currently unavailable outside of \textbf{Ur/Web}.
\item \textbf{Ur/Web} is \textbf{Ur} plus a special standard library,
 plus a special compiler, purpose built for SQL backed web apps.
\item All designed so that well-typed \textbf{Ur/Web} programs don't 'go wrong'.
\end{itemize}
\end{frame}

\section{Badless Double Down}
\subsection{What won't break}
\begin{frame}{Woo!}
\textbf{Ur/Web} applications do not...

\begin{itemize}
\item Suffer from any kinds of code-injection attacks
\item Return invalid HTML
\item Contain dead intra-application links
\item Have mismatches between HTML forms and the fields expected by their handlers
\item Include client-side code that makes incorrect assumptions about the "AJAX"-style services that the remote web server provides
\item Attempt invalid SQL queries
\item Use improper marshaling or unmarshaling in communication with SQL databases or between browsers and web servers
\end{itemize}
\end{frame}


\section{App Demo}
\subsection{Basic app}
\begin{frame}[fragile]{Get on with it...}
As is typical, enough of this jibber-jabber.\\

CUE THE DEMO!\\

It's not "hello, world!", because that's just:
\begin{verbatim}
fun main () = return <xml>Hello, World!</xml>
\end{verbatim}
\end{frame}

\section{Bits}
\subsection{XHTML / HTML}
\begin{frame}{XML Built In}

\begin{itemize}
\item XHTML (and HTML5) are built into the language.\\
\item Structure is also \textbf{verified} by the language.
\end{itemize}
\end{frame}

\begin{frame}[fragile]{XML Built In (yay?)}
Compile time checks of structure:
\begin{verbatim}
<xml><body>
  <h1>Woot</h2> <-- compile error
</body></xml>
\end{verbatim}
Prevents noobing it up from simple errors:
\begin{verbatim}
<xml><body>
  <form action={nomForm}>
		<form> <input type="text"> </form> <-- illegal subform
  </form>
</body></xml>
\end{verbatim}
\end{frame}

\subsection{Query-Tastic}
\begin{frame}{SQL Built In !}

\begin{itemize}
\item Supports Postgresql as default, as well as MySQL, and SQLite.
\item SQL queries are made up of functions as part of the base library.
\item Types for queries and tables are checked at compile time.
\item Database creation SQL is provided by the compiler.
\end{itemize}
\end{frame}
\begin{frame}[fragile]{SQL Built In !}
Make a table:
\begin{verbatim}
table foo : { Id : int, Buzz : string, Created : time }
  PRIMARY KEY Id
\end{verbatim}
Query, just a little bit:
\begin{verbatim}
fun list () =
  rows <- queryX (SELECT * FROM foo)
    (fun row => <xml><div><h1>{[row.Foo.Buzz]}</h1>
        <p>Created: {[row.Foo.Created]}</p>
      </div></xml>);
  return
    <xml>{rows}</xml>
\end{verbatim}
\end{frame}

\subsection{No More JavaScript}
\begin{frame}{Front End McGuffins}
If you didn't have to write the JavaScript, is it really badless JavaScript?\\

\begin{itemize}
\item No difference between back or front end UrWeb code.
\item Compiler works out which code is required where and compiles accordingly.
\item Signal based Functional Reactive Programming.
\item Communication by 'rpc' and typed 'channels'.
\end{itemize}
\end{frame}

\begin{frame}[fragile]{Front End McGuffins}
Dynamic html is simple to include:
\begin{verbatim}
<dyn signal={s <- signal s; doSomethingAmazing s}/>
\end{verbatim}
All your favourite events are there too:
\begin{verbatim}
<button onclick={fn evt => foo evt}/>
\end{verbatim}
\end{frame}

\subsection{More Fun}
\begin{frame}[fragile]{Other Bits}

\begin{itemize}
\item Output is a single executable that can be the web or fastcgi server.
\item The binary produced is extremely efficient compiled C code.
\item The binary does not use garbage collection, relying instead on a technique known as 'region based memory management'.
\item It is "stupid fast". Refer to the Techpower Benchmarks if you're into such things.
\end{itemize}
\end{frame}

\subsection{More?!}
\begin{frame}[fragile]{Bit more}
All requests run inside a `transaction`, analogous to `IO` from Haskell.
\begin{verbatim}
val readBack : transaction int =
  src <- source 0;
  set src 1;
  n <- get src;
  return (n + 1)
\end{verbatim}
EVERYTHING is inside a transaction. Postgres is supported by default due to its strong support for transactions.
\end{frame}

\section{Type System}
\subsection{ML With Bells}
\begin{frame}[fragile]{GIEF TYPES}
\begin{itemize}
\item Type inference
\item Parametric Polymorphism
\end{itemize}
\begin{verbatim} fun id [a] (x : a) : a = x \end{verbatim}
\end{frame}

\subsection{Moar Types}
\begin{frame}[fragile]
Higher order functions:
\begin{verbatim}
fun map [a] [b] (f : a -> b) : list a -> list b
\end{verbatim}
Polymorphic datatypes:
\begin{verbatim}
datatype tree a = Leaf of a | Node of tree a * tree a
\end{verbatim}
\end{frame}

\subsection{Haskelly Bits}
\begin{frame}[fragile]
Typeclasses from Haskell:
\begin{verbatim}
fun max [a] (_ : ord a) (x : a) (y : a) : a = ...
\end{verbatim}
Anonymous Records:
\begin{verbatim}
val x = { A = 0, B = "Fred" }
x.A == 0
\end{verbatim}
\end{frame}

\subsection{First Class}
\begin{frame}[fragile]{Pew-Pew-Pewlymorphism}
"Impredicative or First-Class Polymorphism goes beyond Hidley-Milner's let polymorphism to allow arguments to functions to themselves be polymorphic"
\begin{verbatim}
type nat = t :: Type -> t -> (t -> t) -> t
val zero : nat = fn [t :: Type] (z : t) (s : t -> t) => z
fun succ (n : nat) : nat = 
  fn [t :: Type] (z : t) (s : t -> t) => s (n [t] z s)

val one = succ zero
val two = succ one
val three = succ two

three [int] 0 (plus 1) == 3
\end{verbatim}
\end{frame}

\end{document}