\documentclass{beamer}
\usetheme{Warsaw}
\title[Introduction to UrWeb]{UrWeb\\Functional. Pure. Principled. Eager. Slightly mad.}
\author{Sean Chalmers}
\date{Sun 26 Apr 2015 14:18:00 AEST}
\begin{document}

\begin{frame}
\titlepage
\end{frame}


\begin{frame}{Introduction}
Brief introduction to the Ur language and its current incarnation in Ur/Web.

\begin{itemize}
\item Going Badless
\item Simple app walkthrough
\item Type System
\item Badless - XHTML
\item Badless - SQL
\end{itemize}
\end{frame}

\section{Going Badless}
\subsection{Ur}

\begin{frame}{Ur}
\textbf{Ur} is a relative of Haskell and ML.

\begin{itemize}
\item Pure
\item Statically typed
\item Strict
\item Type level programming
\item Row types
\item Type classes too!
\end{itemize}
\end{frame}

\subsection{Ur/Web}
\begin{frame}{Ur/Web}

\begin{itemize}
\item \textbf{Ur} is currently unavailable outside of \textbf{Ur/Web}.
\vspace{\baselineskip}
\item \textbf{Ur/Web} is \textbf{Ur} plus a special standard library,
 plus a special compiler, purpose built for SQL backed web apps.
\vspace{\baselineskip}
\item All designed so that well-typed \textbf{Ur/Web} programs don't 'go wrong'.
\end{itemize}
\end{frame}

\section{Badless Double Down}
\subsection{What won't break}
\begin{frame}{Woo!}
\textbf{Ur/Web} applications do not...

\begin{itemize}
\item Suffer from any kinds of code-injection attacks
\item Return invalid HTML
\item Contain dead intra-application links
\item Have mismatches between HTML forms and the fields expected by their handlers
\item Include client-side code that makes incorrect assumptions about the "AJAX"-style services that the remote web server provides
\item Attempt invalid SQL queries
\item Use improper marshaling or unmarshaling in communication with SQL databases or between browsers and web servers
\end{itemize}
\end{frame}


\section{App Demo}
\subsection{Basic app}
\begin{frame}{Get on with it...}
As is typical, enough of this jibber-jabber.
\vspace{\baselineskip}
\vspace{\baselineskip}
CUE THE DEMO!
\vspace{\baselineskip}
\vspace{\baselineskip}
It's not "hello, world!", I promise.
\end{frame}










\end{document}